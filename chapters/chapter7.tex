\begin{savequote}[75mm]
%Nulla facilisi. In vel sem. Morbi id urna in diam dignissim feugiat. Proin molestie tortor eu velit. Aliquam erat volutpat. Nullam ultrices, diam tempus vulputate egestas, eros pede varius leo.
%\qauthor{Quoteauthor Lastname}
\end{savequote}

\chapter{Puesta en producción}

\newthought{Para} la puesta en producción del sistema es preciso tomar en cuenta
otras herramientas como el Sistema Operativo donde se ejecutara el sistema,
el Servidor Web que usaremos para desplegar el proyecto, etc.

\section{Servidor Debian Linux}
Debian es una distribución de Linux, un Sistema Operativo Libre con acceso en línea
a los diferentes repositorios que tiene.

Se eligió Debian por el soporte estable que tiene para servidores y por lo practicó
que es al momento de instalar los paquetes de software.

La versión que usaremos para el proyecto es Debian 7.7 que tiene el nombre de
{\it Wheezy}.

\section{Servidor Web Nginx}
Nginx es un Servidor Web ligero y Libre, ademas de eso es de alto rendimiento.
Es conocido por su estabilidad, gran conjunto de características, configuración
simple, y bajo consumo de recursos.
También cuenta con un buen soporte de para FastCGI. Nginx y FastCGI\footnote{FastCGI
es un protocolo para interconectar con un Servidor Web} es una buena elección
al momento de desarrollar con Django y es lo que usaremos para el proyecto,
aunque no es el único camino.

La forma de instalar en Debian es la siguiente:

\begin{verbatim}
# aptitude install nginx
\end{verbatim}

Y la versión que instalará y la que usaremos es la 1.2.1

\section{Base de datos MySQL}
MySQL es un sistema de Base de Datos estable, rápido, multi-usuario y multi-hilo.
La principal meta de MySQL es ser rápido, robusto y fácil de usar.

La forma de instalar es la siguiente:

\begin{verbatim}
# aptitude install mysql-server
\end{verbatim}

Esto nos instalará la versión 5.5 de MySQL.

\section{Instalación y Configuración del Sistema}
Para no tener conflictos de versiones de paquetes de software en el servidor
crearemos un entorno virtual para el proyecto donde ahí instalaremos todos los
paquetes que son requeridos para el funcionamiento del sistema.

Para ello necesitamos instalar en el servidor {\it virtualenv}, esto nos permitirá
crear el entorno virtual que necesitamos para el proyecto, pero antes debemos
instalar {\it python-pip} y una vez hecho esto con {\it pip} instalaremos
{\it virtualenv}.

\begin{verbatim}
# aptitude install python-pip
# pip install virtualenv
\end{verbatim}

\begin{itemize}
  \item {\it pip} - Es un manejador de paquetes de software para Python
  \item {\it virtualenv} - Es una herramienta para mantener las dependencias
    de las versiones de los paquetes de software requeridas por los diferentes
    proyectos y separarlas.
\end{itemize}

Ahora creamos nuestro entorno virtual llamado {\it env} y lo activamos. Y esto
lo hacemos como un usuario normal ya no como {\it root}.

\begin{verbatim}
user:~$ virtualenv env
user:~$ source env/bin/activate
(env)user:~$
\end{verbatim}

En Debian y en otras distribuciones Linux se tiene {\it /var/www/} para los sitios
Web y es ahí donde copiaremos el sistema {\it /var/www/flosite/}.

Pero antes de instalar el sistema crearemos la Base de Datos y un usuario con
privilegios.

\begin{verbatim}
mysql> create database projectdb;
mysql> create user userdb identified by 'xxxxxx';
mysql> grant all privileges on projectdb.* to userdb;
\end{verbatim}

Ingresamos a la carpeta del proyecto pero primero activamos el entorno virtual
que ya creamos anteriormente.

\begin{verbatim}
user:~$ source env/bin/activate
(env)user:~$ cd /var/www/flosite/
(env)user:/var/www/flosite$
\end{verbatim}

Para facilitar la instalación tenemos el siguiente archivo
{\it requirements/project.txt}, ahí se tiene todas las dependencias del proyecto
y lo usamos de la siguiente manera:

\begin{verbatim}
(env)user:/var/www/flosite$ pip install -r requirements/project.txt
\end{verbatim}

Hasta este punto ya tenemos todo lo necesario para que el sistema se ejecute sin
problemas, ahora vamos a sincronizar el proyecto con la Base de Datos para la
creación de las tablas.

Primero configuramos {\it settings.py} para la Base de Datos.

\begin{verbatim}
DATABASES = {
    'default': {
        'ENGINE': 'django.db.backends.mysql',
        'NAME': 'projectdb',
        'USER': 'userdb',
        'PASSWORD': 'xxxxxxx',
        'HOST': '',
        'PORT': '',
    }
}
\end{verbatim}

Una vez editado y guardado {\it settings.py} sincronizamos el proyecto con la
Base de Datos y usamos la instrucción {\it syncdb} para ello.

\begin{verbatim}
(env)user:/var/www/flosite$ python manage.py syncdb
\end{verbatim}

Y para terminar levantamos el sistema:

\begin{verbatim}
(env)user:/var/www/flosite$ python manage.py runserver
\end{verbatim}

En el navegador del servidor ingresamos al URL \url{http://127.0.0.1:8000} y
efectivamente el sistema esta ejecutándose, pero no pueden ingresar desde otras
maquinas, como ya contamos con una IP pública hacemos lo siguiente:

\begin{verbatim}
(env)user:/var/www/flosite$ python manage.py runserver 0.0.0.0:8000
\end{verbatim}

Ahora todos pueden ingresar al sistema usando el IP del servidor como URL
\url{http://xxx.xxx.xxx.xxx:8000}

Pero ejecutar el sistema de esta forma es incomodo y ademas que tenemos que
ejecutarlo manualmente. En la siguiente sección veremos como automatizar esta
tarea en el servidor.

\section{Configuración en el servidor}
Bien, ahora es el turno de Nginx y Debian para el paso final de la puesta en
producción del sistema.

\subsection{Nginx}
Vamos a los archivos de configuración de Nginx en {\it /etc/nginx/} y en
{\it sites-available} creamos el archivo {\it flosite.conf}

\begin{verbatim}
root:~# cd /etc/nginx/sites-available/
root:/etc/nginx/sites-available# touch flosite.conf
\end{verbatim}

En {\it flosite.conf} copiamos las siguientes lineas.

\begin{verbatim}
server {
    listen 80;
    server_name IP.de.nuestro.servidor;
    access_log /var/log/nginx/flosite.access.log;
    error_log /var/log/nginx/flosite.error.log;

    location /static/ { # STATIC_URL
        alias /var/www/flosite/staticfiles/; # STATIC_ROOT
        expires 30d;
    }

    location /static/admin/ { # STATIC_ADMIN
        alias /home/user/env/lib/python2.7/site-packages/django/..../admin/;
        expires 30d;
    }

    location /media/ { # MEDIA_URL
        alias /var/www/flosite/media/; # MEDIA_ROOT
        expires 30d;
    }

    location / {
        include fastcgi_params;
        fastcgi_pass 0.0.0.0:8080;
        fastcgi_split_path_info ^()(.*)$;
        fastcgi_read_timeout    20m;
    }
}
\end{verbatim}

Ahora creamos un enlace simbólico en la carpeta {\it sites-enabled}

\begin{verbatim}
root:/etc/nginx# ln -s sites-available/flosite.conf sites-enabled/
\end{verbatim}

Reiniciamos Nginx para que reconozca la nueva configuración que hemos cargado.

\begin{verbatim}
# service nginx restart
\end{verbatim}

Pero aun nos falta algo, si se fijo en la sección de {\it location} necesitamos
pasarle un fastcgi a nuestra configuración de Nginx, para esto ejecutamos el
siguiente comando:

\begin{verbatim}
(env)user:/var/www/flosite$ python manage.py runfcgi host=0.0.0.0 port=8080
(env)user:/var/www/flosite$
\end{verbatim}

Esto crea el {\it fastcgi} que necesita la configuración que creamos en Nginx.

Si vamos al navegador e ingresamos con nuestra IP, efectivamente esta funcionando
\url{http://ip.de.nuestro.servidor/}

Pero aun hay un comando que hay que escribir manualmente. Ahora es el turno de
Debian Linux.

\subsection{Configuración en el Servidor Debian Linux}
Para automatizar la tarea de levantar el sistema al momento de iniciar el servidor
necesitamos usar {\it update-rc.d} que es el comando usado para decidir la secuencia
y el runlevel de ejecución del script que crearemos.

Creamos nuestro script {\it flosite} en {\it /etc/init.d/}

\begin{verbatim}
# touch /etc/init.d/flosite
\end{verbatim}

Ahora copiamos las lineas a continuación en el archivo creado.

\begin{verbatim}
#! /bin/bash
#
### BEGIN INIT INFO
# Provides:          flosite
# Required-Start:    mysql
# Required-Stop:     mysql
# Default-Start:     2 3 4 5
# Default-Stop:      0 1 6
# Short-Description: Inicia el sistema de flosite
# Description:       Este script fue escrito por Benjamin Perez
### END INIT INFO

# Este script es para levantar el sistema de seguridad industrial
echo "This part always gets executed"

case "$1" in
  start)
    echo "Starting flosite"
    echo "Please waith ....."
    source /home/user/env/bin/activate
    cd /var/www/flosite
    python manage.py runfcgi host=0.0.0.0 port=8080
    deactivate
    echo "Done" 
    ;;
  stop)
    echo "Stopping flosite"
    echo "Please waith ..."
    # TODO
    echo "Done"
    ;;
  restart)
    echo "Restarting flosite"
    echo "Please waith ....."
    # TODO
    ;;
  *)
    echo "Usage: /etc/init.d/flobasic {start|stop|restart}"
    exit 1
    ;;
esac

exit 0

\end{verbatim}

En este script algo que es muy importante son estas lineas:

\begin{verbatim}
# Required-Start:    mysql
# Required-Stop:     mysql
\end{verbatim}

Con estas lineas le decimos que nuestro script se ejecute o se detenga una vez
que ya inicio o se detuvo MySQL. Esto es importante por que si el script se
ejecuta antes de MySQL nos dará error y no podremos ingresar al sistema.

Ahora le damos permisos de ejecución:

\begin{verbatim}
# chmod +x /etc/init.d/flosite
\end{verbatim}

Finalmente ejecutamos {\it update-rc.d} para que el script se ejecute automáticamente
al iniciar el servidor.

\begin{verbatim}
# update-rc.d /etc/init.d/flosite defaults
\end{verbatim}

Al usar la opción {\it defaults} update-rc.d creará enlaces para iniciar el servicio
en los runlevels 2345 y detendrá el servicio en los runlevels 016.

Los {\it runlevels} básicamente se refieren a los diferentes niveles que tiene
un Servidor Linux para iniciar y parar servicios.

Con esto ya tenemos un servidor listo para ejecutar automáticamente el proyecto.

