\begin{savequote}[75mm]
%Nulla facilisi. In vel sem. Morbi id urna in diam dignissim feugiat. Proin molestie tortor eu velit. Aliquam erat volutpat. Nullam ultrices, diam tempus vulputate egestas, eros pede varius leo.
%\qauthor{Quoteauthor Lastname}
\end{savequote}

\chapter{Pruebas del Sistema}

\newthought{En} el capítulo tres vimos como la metodología BDD es una herramienta útil que
fomenta la colaboración entre desarrolladores, testers y clientes. Entendimos
como BDD es una técnica de desarrollo ágil de software.

En el capítulo cuatro vimos la aplicación de BDD en el proyecto e hicimos los
escenarios respectivos para cumplir los requerimientos de funcionalidad del
sistema.

\begin{itemize}
    \item Registro de Vehículos
    \item Registro de Ingreso de personas
    \item Registro de mantenimiento
    \item Reportes
    \item Notificaciones
\end{itemize}

En este capítulo mostraremos como esos escenarios que ya definimos nos sirven
para hacer las pruebas finales.

Los escenarios y el código que desarrollamos para que pasen los tests eran solo
una muestra de como el sistema iba a cumplir todos los requerimientos de
funcionalidad.

Hasta este punto del proyecto ya se tiene el sistema completo. Y ahora es cuando
usaremos esos mismos escenarios para mostrar como el sistema cumple con las
funcionalidades exigidas.

Para realizar estas pruebas necesitaremos de dos herramientas mas en el
entorno virtual de desarrollo.

\begin{itemize}
  \item django-behave\footnote{Django-Behave, desarrollado por \citeauthor*{website:djangobehave}}
  \item splinter\footnote{Splinter, desarrollado por \citeauthor{website:splinter}}
\end{itemize}

\section {django-behave}
Es un TestRunner de Django para el modulo de Behave BDD, ya en el capítulo tres
y cuatro vimos como trabaja {\it behave} con BDD.

\subsection{Instalación}
Para la instalación primero entraremos a nuestro entorno virtual de desarrollo y
usando {\it pip}\footnote{PIP, es un manejador de paquetes de Python} lo
instalaremos.
\begin{verbatim}
project$ source ~/env/bin/active
(env)project$ pip install django-behave
\end{verbatim}

\subsection{Configuración}
Una vez instalado django-behave pasamos a hacer la configuración en el archivo
{\it settings.py} y hacemos los cambios como se muestra a continuación.
\begin{verbatim}
INSTALLED_APPS = (
    ..
    'django_behave',
)

TEST_RUNNER = 'django_behave.runner.DjangoBehaveTestSuiteRunner'
\end{verbatim}

\section {splinter}
Splinter\footnote{Splinter, \url{http://splinter.cobrateam.info/en/latest/},
consultado en noviembre de 2014} es una herramienta Open Source para hacer
pruebas en aplicaciones web usando Python. Permite automatizar acciones dentro
del navegador, como visitar URLs e interactuar con los items.

\subsection{Instalación}
Para instalarlo usaremos {\it pip} como en los casos anteriores y siempre
trabajando dentro de nuestro entorno virtual:

\begin{verbatim}
(env)project$ pip install splinter
\end{verbatim}

Enhorabuena, ya tenemos instalado {\it django-behave} y {\it splinter}, ahora la
forma de usarlo es de la siguiente manera:

\begin{verbatim}
(env)project$ python manage.py test app
\end{verbatim}

Como ya tenemos configurado django-behave en {\it settings.py} el archivo
{\it manage.py} reconoce la orden {\it test} y lo que viene después es el
nombre de la aplicación que en el ejemplo solo se lo llama {\it app}.

Pero {\it splinter} no se ejecuta de manera mágica, necesitamos decirle a los
{\it features} de BDD de cada aplicación de nuestro proyecto que debe usar
splinter al momento de ejecutar los tests.

\section{Configuración de los escenarios}
En el capítulo cuatro pusimos todos los escenarios en una misma carpeta {\it features}
y cuando lo ejecutábamos con {\it behave} recorría todos los escenarios.

Pero ahora, como ya tenemos el sistema completo con sus aplicaciones respectivas
para cada funcionalidad vamos a llevar los escenarios a la aplicación que le
corresponde, veamos el caso de {\it accounts}.

\begin{verbatim}
flosite
    |-- manage.py
    |-- flosite (carpeta principal)
    |   `-- ...
    |-- accounts/
    |  |-- features
    |  |   |-- behave.ini
    |  |   |-- environment.py
    |  |   |-- login.feature
    |  |   `-- steps
    |  |       `-- step_login.py
    |  |-- __init__.py
    |  |-- models.py
    |  |-- urls.py
    |  `-- views.py
    |-- ...
\end{verbatim}

En la aplicación {\it accounts} ahora tenemos la carpeta {\it features} y es ahí
donde están los archivos de Configuración y su escenario {\it login.feature},
veamos {\it behave.ini}:

\begin{verbatim}
[behave]
lang = es
\end{verbatim}

En este archivo le decimos de manera explicita que los escenarios están escritos
en {\it Español}.

El archivo {\it environment.py} es el que mas nos interesa ahora, porque es ahí
donde le decimos a {\it django-behave} que use splinter al momento realizar los
tests.

\begin{verbatim}
from splinter.browser import Browser

def before_all(context):
    context.browser = Browser()

def after_all(context):
    context.browser.quit()
    context.browser = None
\end{verbatim}

El método {\it before\_all}, se encarga de que antes de cualquier cosa inicialice
el Browser o el navegador.

El método {\it after\_all}, se encarga de que después que realice todos los pasos
cierre el navegador.

\section{Escenario Account}
Como ya mencionamos, trabajaremos con {\it accounts} para ver como se ejecuta
el test con splinter paso a paso. Veamos el archivo {\it login.features}:

{\scriptsize
\begin{minted}{Gherkin}
# language: es

Característica: Pagina principal
    Como un guardia
    Quiero ingresar al sistema
    Para registrar ingresos

    Esquema del escenario: Iniciar sesión
        Dado que ingreso al sistema con el url "http://127.0.0.1:8000/"
        Y voy a la opcion "Ingresar"
        Y entro con mi nombre de usuario <username>
        Y mi contraseña <password>
        Cuando oprima el boton "Ingresa"
        Entonces ingreso al sistema y leo el mensaje "Bienvenido"

        Ejemplos:
            | username | password |
            | benjamin | asdf     |
\end{minted}
}

A diferencia de los escenarios que ya vimos, aquí le decimos que tiene que entrar
a una dirección URL.

Una vez hecho esto en la carpeta {\bfseries steps} escribimos el código necesario
para este escenario en el archivo {\it step\_login.py}.

\begin{verbatim}
# encoding: utf-8

from behave import given, when, then

@given(u'que ingreso al sistema con el url "{url}"')
def impl(context, url):
    br = context.browser
    br.visit(url)

@given(u'voy a la opcion "{option}"')
def impl(context, option):
    link = context.browser.find_link_by_text(option).first
    assert link
    link.click()

@given(u'entro con mi nombre de usuario {usr}')
def impl(context, usr):
    context.browser.fill('identification', usr)
    context.user = usr
    assert usr

@given(u'mi contraseña {pwd}')
def impl(context, pwd):
    context.browser.fill('password', pwd)
    context.password = pwd
    assert pwd

@when(u'oprima el boton "{btn}"')
def impl(context, btn):
    button = context.browser.find_by_value(btn).first
    assert button
    button.click()

@then(u'ingreso al sistema y leo el mensaje "{msg}"')
def impl(context, msg):
    assert msg in context.browser.html
\end{verbatim}

Cada método recibe como atributo a {\it context} y como en el archivo
{\it environment.py} ya lo inicializamos con splinter, solo tenemos que llamar
a sus métodos e ir recorriendo la pagina siguiendo los pasos escritos
en {\it login.features}.

Una vez escrito y configurado todo esto ejecutamos el test:

\begin{verbatim}
(env)project$ python manage.py test accounts
\end{verbatim}

Cuando se ejecuta el test llama a {\it splinter} que a la vez ejecuta un navegador
de pruebas donde recorre los pasos que hemos definido en los escenarios.

Esto podemos verlo en el siguiente vídeo que es de mi autoría puesto en youtube.com
con el siguiente título {\it BDD, django-behave and splinter}\footnote{BDD,
django-behave and splinter: publicado el 2 de diciembre de 2014} en la URL: \url{http://youtu.be/KT6DCFfCFnY}

