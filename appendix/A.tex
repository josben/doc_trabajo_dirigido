
\chapter{Instalación y configuración de herramientas}

\newpage

\section{Entorno Virtual}
Para crearnos un entorno virtual de desarrollo que no afecte en las dependencias
de versiones de otros programas usaremos {\it virtualenv} y para esto seguimos
los siguientes pasos como {\it root}.

\begin{verbatim}
# aptitude install python-pip
# pip install virtualenv
\end{verbatim}

\begin{itemize}
  \item {\it pip} - Es un manejador de paquetes de software para Python
  \item {\it virtualenv} - Es una herramienta para mantener las dependencias
    requeridas por los diferentes proyectos y separarlas.
\end{itemize}

Ahora creamos nuestro entorno virtual llamado {\it env} y lo activamos. Y esto
lo hacemos como un usuario normal ya no como {\it root}.

\begin{verbatim}
user:~$ virtualenv env
user:~$ source env/bin/activate
(env)user:~$
\end{verbatim}

Note que delante de nuestro nombre de usuario aparece el nombre de nuestro
entorno virtual: {\it (env)}

\section{Behave}
Para instalar {\it behave} lo haremos usando {\it pip}, pero lo haremos dentro
de nuestro entorno virtual de desarrollo.

\begin{verbatim}
(env)user:~$ pip install behave
\end{verbatim}

Con esto ya tenemos instalado {\it behave} y lo podemos usar de la siguiente manera:

\begin{verbatim}
(env)user:~/project/features$ behave registro.feature
\end{verbatim}

\section{Django-Behave}
Para la instalación primero entraremos a nuestro entorno virtual de desarrollo y
usando {\it pip} lo instalaremos.

\begin{verbatim}
user:~ $ source env/bin/active
(env)user:~ $ pip install django-behave
\end{verbatim}

\subsection{Configuración}
Una vez instalado django-behave pasamos a hacer la configuración en el archivo
{\it settings.py} y hacemos los cambios como se muestra a continuación.
\begin{verbatim}
INSTALLED_APPS = (
    ..
    'django_behave',
)

TEST_RUNNER = 'django_behave.runner.DjangoBehaveTestSuiteRunner'
\end{verbatim}



\section{Splinter}
Para instalarlo usaremos {\it pip} como en los casos anteriores y siempre
trabajando dentro de nuestro entorno virtual:

\begin{verbatim}
(env)user:~ $ pip install splinter
\end{verbatim}

El archivo {\it environment.py} que se encuentra en la carpeta {\it features}
de cada {\it app} del proyecto es donde escribimos el siguiente código,
porque es ahí donde le decimos a {\it django-behave} que use splinter al momento
realizar los tests.

\begin{verbatim}
from splinter.browser import Browser

def before_all(context):
    context.browser = Browser()

def after_all(context):
    context.browser.quit()
    context.browser = None
\end{verbatim}

La forma de usarlo es de la siguiente manera:

\begin{verbatim}
(env)project$ python manage.py test app
\end{verbatim}

