
\chapter{Tests de Ejecución}

\newpage

En el capitulo 4 vimos los escenarios de funcionalidad del sistema, en este
anexo mostraremos los tests escritos en Python que validan esas funcionalidades.

\section{Steps - Antes del desarrollo del sistema}
El código que se muestra a continuación valida los escenarios escritos antes
de que el sistema fuera desarrollado para mostrar como el sistema cumplirá con
las funcionalidades requeridas.

{\scriptsize
  \begin{verbatim}
# encoding: utf-8

from behave import given, when, then
from hamcrest import assert_that, equal_to

class LoginUser(object):
    def __init__(self, uname=None, pwd=None, button=None):
        self.uname = uname
        self.pwd = pwd
        self.button = button

    def login(self):
        assert self.uname is not None
        assert self.pwd is not None
        assert self.button is not None

        if self.button == 'iniciar_sesion':
            return 'Iniciar sesion'
        else:
            return 'Error al ingresar'

class Person(object):
    def __init__(self, type_doc=None, num_doc=None,
                 name=None, city=None, reason=None):
        self.type_doc = type_doc
        self.num_doc = num_doc
        self.name = name
        self.city = city
        self.reason = reason

    def register_person(self):
        assert self.type_doc is not None
        assert self.num_doc is not None
        assert self.name is not None
        assert self.city is not None
        assert self.reason is not None

        return 'Persona registrada'
class RegisterCar(object):
    def __init__(self, internal_number=None, parqueo_out=None,
                 parqueo_in=None, item_out=None, item_in=None,
            km_out=None, km_in=None, escaleras_out=None, escaleras_in=None,
            date_out=None, date_in=None, time_out=None, time_in=None):
        self.internal_number = internal_number
        self.parqueo_out = parqueo_out
        self.parqueo_in = parqueo_in
        self.item_out = item_out
        self.item_in = item_in
        self.km_out = km_out
        self.km_in = km_in
        self.escaleras_out = escaleras_out
        self.escaleras_in = escaleras_in
        self.date_out = date_out
        self.date_in = date_in
        self.time_out = time_out
        self.time_in = time_in
        self.reg_out = False
        self.reg_in = False

    def validar_salida(self):
        assert self.internal_number is not None
        assert self.parqueo_out is not None
        assert self.item_out is not None
        assert self.km_out is not None
        assert self.escaleras_out is not None
        assert self.date_out is not None
        assert self.time_out is not None

        self.reg_out = True

        return 'Registro guardado'

# ##########################
# Steps: Ingreso al sistema
# ##########################

@given(u'que he introducido mi nombre de usuario {username}')
def step_impl(context, username):
    context.username = username

@given(u'mi contraseña {password}')
def step_impl(context, password):
    context.password = password

@when(u'oprima el boton {button}')
def step_impl(context, button):
    context.button = button
    context.login = LoginUser(context.username,
                              context.password, context.button)

@then(u'debo ingresar al sistema')
def step_impl(context):
    assert_that('Iniciar sesion', equal_to(context.login.login()))

# ##########################
# Steps: Registro de visitas
# ##########################

@given(u'que ingresa una persona con documento {type_doc}')
def step_impl(context, type_doc):
    context.type_doc = type_doc
@given(u'con numero de docuento {num_doc}')
def step_impl(context, num_doc):
    context.num_doc = num_doc

@given(u'con el siguiente nombre {name}')
def step_impl(context, name):
    context.name = name

@given(u'proviene de la ciudad {city}')
def step_impl(context, city):
    context.city = city

@given(u'con el siguinte motivo {reason}')
def step_impl(context, reason):
    context.reason = reason

@when(u'registre a la persona')
def step_impl(context):
    context.person = Person(context.type_doc,
                            context.num_doc,
                            context.name,
                            context.city,
                            context.reason)

@when(u'el {num_doc} sea valido')
def step_impl(context, num_doc):
    assert_that(num_doc, equal_to(context.num_doc))

@then(u'regitro a la persona {name} con documento {num_doc}')
def step_impl(context, name, num_doc):
    assert_that('Persona registrada',
                equal_to(context.person.register_person()))

# ######################################
# Steps: Registro de salida de vehiculos
# ######################################

@given(u'que sale el vehiculo con numero interno {internal_number}')
def step_impl(context, internal_number):
    context.internal_number = internal_number
@given(u'sale del parqueo {parqueo_out}')
def step_impl(context, parqueo_out):
    context.parqueo_out = parqueo_out

@given(u'con el conductor de salida con item {item_out}')
def step_impl(context, item_out):
    context.item_out = item_out

@given(u'con kilometraje de salida {km_out}')
def step_impl(context, km_out):
    context.km_out = km_out

@given(u'con las escaleras {escaleras_out} de salida')
def step_impl(context, escaleras_out):
    context.escaleras_out = escaleras_out

@given(u'con fecha {date_out} de salida')
def step_impl(context, date_out):
    context.date_out = date_out

@given(u'con hora {time_out} de salida')
def step_impl(context, time_out):
    context.time_out = time_out

@when(u'registre salida del vehiculo {internal_number}')
def step_impl(context, internal_number):
    context.register_car = RegisterCar(internal_number=context.internal_number,
                                       parqueo_out=context.parqueo_out,
                                       item_out=context.item_out,
                                       km_out=context.km_out,
                                       escaleras_out=context.escaleras_out,
                                       date_out=context.date_out,
                                       time_out=context.time_out)
    context.car_list = []
@when(u'los datos de salida sean validos')
def step_impl(context):
    assert_that('Registro guardado',
                 equal_to(context.register_car.validar_salida()))

# #######################################
# Steps: Registro de retorno de vehiculos
# #######################################

@given(u'que retorna el vehiculo con numero interno {internal_number}')
def step_impl(context, internal_number):
    for c in context.car_list:
        if c.internal_number == internal_number:
            context.car_tmp = c

@given(u'retorna al parqueo {parqueo_in}')
def step_impl(context, parqueo_in):
    context.car_tmp.parqueo_in = parqueo_in

@given(u'con el conductor de retorno con item {item_in}')
def step_impl(context, item_in):
    context.car_tmp.item_in = item_in

@given(u'con kilometraje de retorno {km_in}')
def step_impl(context, km_in):
    context.car_tmp.km_in = km_in

@given(u'con las escaleras {escaleras_in} de retorno')
def step_impl(context, escaleras_in):
    context.car_tmp.escaleras_in = escaleras_in

@given(u'con fecha {date_in} de retorno')
def step_impl(context, date_in):
    context.car_tmp.date_in = date_in

@given(u'con hora {time_in} de retorno')
def step_impl(context, time_in):
    context.car_tmp.time_in = time_in
@when(u'registre retorno del vehiculo {internal_number}')
def step_impl(context, internal_number):
    assert False

@when(u'los datos del kilometraje sean correctos')
def step_impl(context):
    assert False

@when(u'las escaleras de retorno sean las mismas')
def step_impl(context):
    assert False

@when(u'el conductor sea el mismo')
def step_impl(context):
    assert False

# ###################################
# Steps para ambos: salida y retorno
# ###################################

@when(u'se validen los datos de salida')
def step_impl(context):
    assert_that('Registro guardado',
    equal_to(context.register_car.validar_salida()))

@then(u'registro al vehiculo {internal_number}')
def step_impl(context, internal_number):
    assert context.register_car.reg_out
    context.car_list.append(context.register_car)

  \end{verbatim}
}

Para que el código anterior funcione correctamente es necesario inicializar
algunas variables en {\it environment.py}

{\scriptsize
  \begin{verbatim}
import logging

def before_all(context):
    context.register_car = None
    context.parqueo_out = None
    context.km_out = None
    context.escaleras_out = None
    context.item_out = None
    context.date_out = None
    context.time_out = None
    context.dict_cars = {}
    context.warning = False

    if not context.config.log_capture:
        logging.basicConfig(level=logging.DEBUG)
  \end{verbatim}
}

\section{Steps - Con el sistema concluido}
Una vez concluido el Sistema usamos los mismos escenarios que se plantearon en
el Capitulo 4 para las pruebas finales, con la diferencia de que ahora lo probaremos
en el sistema.

{\scriptsize
  \begin{verbatim}
# encoding: utf-8

from behave import given, when, then

# ##################################
# Steps: Para el ingreso al Sistema
# ##################################

@given(u'que ingreso al sistema con el url "{url}"')
def impl(context, url):
    br = context.browser
    br.visit(url)

@given(u'voy a la opcion "{option}"')
def impl(context, option):
    link = context.browser.find_link_by_text(option).first
    assert link
    link.click()
#    assert option in context.browser.html
#    msg = context.browser.find_link_by_partial_text(option).first
#    assert msg

@given(u'entro con mi nombre de usuario {usr}')
def impl(context, usr):
    context.browser.fill('identification', usr)
    context.user = usr
    assert usr

@given(u'mi contraseña {pwd}')
def impl(context, pwd):
    context.browser.fill('password', pwd)
    context.password = pwd
    assert pwd

@when(u'oprima el boton "{btn}"')
def impl(context, btn):
    button = context.browser.find_by_value(btn).first
    assert button
    button.click()

@then(u'ingreso al sistema y leo el mensaje "{msg}"')
def impl(context, msg):
    assert msg in context.browser.html

# ####################################
# Steps: Para el registro de Vehiculos
# ####################################

@given(u'que ingreso al sistema con el url "{url}"')
def impl(context, url):
    br = context.browser
    br.visit(url)

@given(u'voy a la opcion "{option}"')
def impl(context, option):
    link = context.browser.find_link_by_text(option).first
    assert link
    link.click()
#    assert option in context.browser.html
#    msg = context.browser.find_link_by_partial_text(option).first
#    assert msg

@given(u'entro con mi nombre de usuario {usr}')
def impl(context, usr):
    context.browser.fill('identification', usr)
    context.user = usr
    assert usr

@given(u'mi contraseña {pwd}')
def impl(context, pwd):
    context.browser.fill('password', pwd)
    context.password = pwd
    assert pwd

@when(u'oprima el boton "{btn}"')
def impl(context, btn):
    button = context.browser.find_by_value(btn).first
    assert button
    button.click()

@then(u'ingreso al sistema y leo el mensaje "{msg}"')
def impl(context, msg):
    assert msg in context.browser.html

@given(u'que vamos a la pagina de registro "{url}"')
def impl(context, url):
    br = context.browser
    br.visit(url)

@given(u'registra el vehiculo con numero interno {car} esta de {status}')
def impl(context, car, status):
    context.browser.fill('car', car)
    context.car = car
    context.status = status

@given(u'sale/retorna del parqueo {parking}')
def impl(context, parking):
    context.parking = parking

@given(u'con el conductor con item {driver}')
def impl(context, driver):
    context.browser.fill('employee', '')
    context.browser.fill('employee', driver)
    context.driver = driver

@given(u'tiene el kilometraje {km}')
def impl(context, km):
    if context.status == 'salida':
        pass
    else:
        current_km = context.browser.find_by_name('km').first.value
        km = int(current_km) + 25
        context.browser.fill('km', '')
        context.browser.fill('km', km)

@given(u'con las escaleras {ladders}')
def impl(context, ladders):
    context.browser.fill('ladders', '')
    context.browser.fill('ladders', ladders)

@given(u'en fecha {date}')
def impl(context, date):
    link = context.browser.find_by_id('change_date').first
    assert link
    link.click()
    context.browser.fill('register_date', '')
    context.browser.fill('register_date', date)

@given(u'en horas {time}')
def impl(context, time):
    context.browser.fill('time', '')
    context.browser.fill('time', time)

@when(u'registre al vehiculo {car} con su estado de {status}')
def impl(context, car, status):
    button = context.browser.find_by_value("Registrar").first
    assert button
    button.click()

@when(u'lea el mensaje "{msg}"')
def impl(context, msg):
    assert msg in context.browser.html

@then(u'guardo el registro del vehiculo {car} correctamente')
def impl(context, car):
    pass

  \end{verbatim}
}

Para que este código funcione correctamente es necesario también inicializar
algunas variables en {\it environment.py}

{\scriptsize
  \begin{verbatim}
from splinter.browser import Browser

def before_all(context):
    context.browser = Browser()

def after_all(context):
    context.browser.quit()
    context.browser = None
  \end{verbatim}
}
